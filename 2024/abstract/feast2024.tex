\documentclass[sigconf]{acmart}
\usepackage{microtype}
\usepackage{url}
\usepackage{enumitem}

% This block is copied from the e-mail you get after completing the eRights form
\copyrightyear{2024}
\acmYear{2024}
\setcopyright{rightsretained}
\acmConference[CCS '24]{Proceedings of the 2024 ACM SIGSAC Conference on Computer and Communications Security}{October 14--18, 2024}{Salt Lake City, UT, USA}
\acmBooktitle{Proceedings of the 2024 ACM SIGSAC Conference on Computer and Communications Security (CCS '24), October 14--18, 2024, Salt Lake City, UT, USA}
\acmDOI{10.1145/3658644.3691553}
\acmISBN{979-8-4007-0636-3/24/10}

\settopmatter{printacmref=true}

\pdfadjustspacing=2

\newcommand\bio[1]{\noindent\textbf{#1}}

\begin{document}
\fancyhead{}

\title[FEAST'24]{FEAST'24: Sixth Workshop on Forming an Ecosystem Around Software Transformation}

\author{Ryan Craven}
\affiliation{%
  \institution{Office of Naval Research}
  \city{Arlington}
  \state{Virginia}
  \country{USA}
}
\email{ryan@rcraven.net}

\author{Matthew Mickelson}
\affiliation{%
  \institution{MITRE}
  \city{McLean}
  \state{Virginia}
  \country{USA}
}
\email{mmickelson@mitre.org}

% Required for even page headers
\renewcommand{\shortauthors}{Ryan Craven and Matthew Mickelson}

\begin{abstract}
The Sixth Workshop on Forming an Ecosystem Around Software Transformation (FEAST)
revives the series, with the original five events taking place from 2016-2020.
FEAST is concerned with all aspects of achieving effective, robust, and appraisable late-stage
transformation of software for security. Late-stage transformations allow
third parties to deeply tailor existing software to their mission,
customizing it with little to no access to source code or support from the original developer.

Research has shown that late-stage software customization is of particular benefit to security-conscious
software consumers who must use closed-source or source-free binary software components in mission-critical settings,
or who must harden software against newly emerging attacks not anticipated during the software's original design and development.
However, there is still a long way to go toward achieving sound and robust transformations whose
holistic benefits to deployed software are fully appraisable.
Motivated by these outstanding challenges, FEAST continues in its goal to form an active ecosystem of strategies and tools
for accomplishing source-free binary code transformation reliably and on-demand.
\end{abstract}

% CCS Concepts code generated from https://dl.acm.org/ccs
\begin{CCSXML}
  <ccs2012>
  <concept>
  <concept_id>10002978.10003022</concept_id>
  <concept_desc>Security and privacy~Software and application security</concept_desc>
  <concept_significance>500</concept_significance>
  </concept>
  <concept>
  <concept_id>10011007.10011074.10011111</concept_id>
  <concept_desc>Software and its engineering~Software post-development issues</concept_desc>
  <concept_significance>500</concept_significance>
  </concept>
  </ccs2012>
\end{CCSXML}
  
\ccsdesc[500]{Security and privacy~Software and application security}
\ccsdesc[500]{Software and its engineering~Software post-development issues}

% Separating your keywords; with semi-colons; is mandatory
\keywords{binary software; software debloating; software de-layering; software security hardening; binary rewriting; software transformation}

\maketitle

\section{Introduction}
In 2024, software is more bloated than ever~\cite{hubert2024}.  Since the
most recent FEAST in 2020, software size has continued to climb at an
exponential rate. The proliferation of "smart" features into everyday consumer
products has carried along with it a need for manufacturers across all
industries to rapidly ship complex software functionality out to a host of
cost-sensitive products. Nowhere has the change been more dramatic than the auto
industry, where even low-end vehicles can have 100 ECUs and 100 million of lines
of code, and advances in autonomous driving functions hold the potential for
some new cars to push the 500 million lines of code mark~\cite{charette2021}.
That would place an amount of code equal to one quarter the size of all the
software running Google's \emph{entire} internet services catalog ten years
ago~\cite{potvin2015}, into every family vehicle.

While economics have always driven developers to ship software with lots of
features for broad appeal, it is the advancement of efficient and easy-to-use
code reuse practices along with the powerful gains they deliver to programmer
productivity that enable the explosive growth we see.  We find it important to
stress that maximizing code reuse is not itself a bad thing: Consumers and
businesses benefit from a rapid pace of less expensive, more feature-rich
products, and developers spend their limited time more efficiently.  But there
is a growing problem that security-conscious consumers know all too well: Most
of the code in any modern system is unnecessary or even potentially undesirable
to its users~\cite{quach2017}.  One study found that, on average, only 10\% of
the functions in the most frequently used shared libraries in Ubuntu are ever
invoked by common programs~\cite{quach2018}.

Identifying when unneeded or undesirable code is being added to a system is
difficult, due in large part to an enormous amount of complexity that gets
hidden behind slick abstractions: frameworks, middlewares, container
orchestration, and so many libraries that today's popular languages all need
their own package managers just to make things manageable~\cite{dusing2022}.  We
expect these conditions will persist, as the cost of the complexity seems cheap
compared to the value gained. The costs, however, are only thought to be cheap
because the market measures them at \emph{development time}, where all the
benefits (increased productivity resulting from the code reuse and abstractions)
are being gained.

The follow-on costs (increased maintenance, features that later become
bugs, expanded attack surfaces, and opaque software supply chains) are difficult
to capture, and varyingly affect different segments of consumers.
FEAST is a recognition of these costs.  
Rather than fight market forces directly, which is unlikely to be effective, the
FEAST Workshop is devoted to improving the feasibility and effectiveness of
\emph{late-stage software transformation}. Late-stage transformations modify
low-level software after it has been designed, developed, and compiled into a
distributable product. Such technologies offer consumers the ability to
customize software to their particular requirements, such as by removing
unneeded features, stripping out unnecessary complexity, or adding hardened
security defenses against dangerous attacks. Source-free software transformation
challenges of particular interest include:
\begin{itemize}[itemsep=.5ex plus1pt minus0pt]
  \item \textbf{software debloating}, which concerns the removal of software behaviors, code, or data that is unneccessary for a given consumer's needs;
  \item \textbf{software de-layering}, which removes levels of indirection or abstraction layers that impede efficiency;
  \item \textbf{software security hardening}, which concerns adding extra security checks and other defenses to code in order to thwart attacks;
  \item \textbf{post-deployment patching}, which allows binary code to be more easily modified to replace or remove functionalities;
  \item \textbf{attack surface discovery and reduction}, which discovers and mitigates potential opportunities for abuse and compromise of binary software products;
  \item \textbf{software self-healing}, which transforms software to detect and remediate faults unanticipated by its authors;
  \item \textbf{transformation-aware reverse-engineering}, which lifts low-level software to a higher-level form amenable to analysis, transformed, and then lowered back to executable form without sacrificing efficiency; and
  \item \textbf{low-level formal methods}, which extend automated theorem proving, model-checking, and type-based verification typically used at the source level for high assurance code down to executable binaries.
\end{itemize}

The goal of FEAST is to cultivate a robust ecosystem of these and other
technologies relevant to practical, effective customization of binary software
without the aid of source code or developer support.

\subsection{The Continuing Need for FEAST}
Since the most recent FEAST in 2020, numerous major events continue to motivate our vision.
In 2021, an obscure feature added to the Log4j library
almost eight years prior triggered a global cybersecurity emergency~\cite{cisalog4j}.  The cause was
due to one user of the library adding a feature (support for JNDI lookups) to
make their life more convenient, and their patch was committed by the
maintainers less than 24 hours later~\cite{jndiflaw}.

Seeing the lack of
scrutiny going into widespread code reuse, attackers began more
heavily targeting overworked package maintainers~\cite{duan2021}, leading to a
coordinated multi-year operation against the maintainer of XZ Utils compression
library~\cite{xzutils}. The Sunburst malware was discovered to have been hiding
in Solarwinds Orion, a software product intended to \emph{improve} security, for months undetected~\cite{sunburst}. And
unnecessary support for obscure image formats led to a zero-click vulnerability
in iMessage~\cite{forcedentry}.

The rapid and pervasive adoption of AI coding assistants is already having broad
effects on how software is built and deployed~\cite{klemmer2024}.  One
possibility is that complexity and opacity increase another level as the new technology
leads to changes in behaviors.  For instance, LLM hallucinations of package
library names were observed being used in the wild~\cite{lanyado2024}. As we
progress toward FEAST's goal of practical and effective customization of binary
software, consumers that bear a larger burden from the offset costs of
increasing size and complexity will benefit as they become empowered to exert
more rigor over and reshape the software they deploy.

\section{Sixth Workshop Program}
The sixth FEAST workshop consists of four full paper presentations, and ten talk
proposal presentations.  Only full papers were entered into official proceedings.
The talk proposal was a lighter-lift submission type we created 
to incorporate more diverse perspectives.  The acceptance rate was 100\%,
as all submissions were deemed high quality and relevant to scope.
The presentations are organized into four sessions, each with a different theme
tied to a property about late-stage transformations that we seek to improve:
Soundness, Robustness, Appraisability, and Enabling Technology.

\section{Workshop Organization}
The following program committee members helped organize the 2024 FEAST Workshop:
\begin{itemize}
  \item Ryan Craven (Office of Naval Research)
  \item Matthew Mickelson (MITRE)
  \item Jason Li (Trusted ST)
  \item Sukarno Mertoguno (Georgia Tech)
  \item Daniel Koller (Pennsylvania State University)
  \item Nathan Burow (MIT Lincoln Laboratory)
\end{itemize}

\section{Acknowledgments}
The workshop chairs for FEAST'24 wish to thank all authors for submitting papers,
as well as past chairs
Taesoo Kim, Dinghao Wu,
Yan Shoshitaishvili, Mayur Naik,
Adam Doup\'e, Zhiqiang Lin,
Long Lu, and Kevin Hamlen
for their stewardship of the FEAST community over the years.

\bibliographystyle{ACM-Reference-Format}
\balance
\bibliography{references}

\end{document}

